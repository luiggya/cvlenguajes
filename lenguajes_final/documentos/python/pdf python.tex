\documentstyle[11pt]{report}
%\setcounter{page}{6}

\oddsidemargin 0pt \evensidemargin 0pt
\topmargin=1.25in
\headheight 10pt \headsep 10pt \footheight 10pt \footskip 24pt
\textheight 10in \textwidth 6.5in \columnsep 10pt \columnseprule 0pt

\font\namefont=cmr10 scaled\magstep2
\voffset=-.75in
\parskip=.075in
\parindent=0in

\thispagestyle{empty}
\begin{document}
\bigskip



\bigskip
\large \centerline {\namefont \bf DOCUMENTO DE ESPECIFICACIONES, OBSERVACIONES, CONCLUSIONES} 
\large \centerline {\namefont \bf Y EXPERIENCIAS DEL DESARROLLO DE NUESTRA}
\large \centerline {\namefont \bf APLICACION PYTHON}
\bigskip

\centerline{\namefont  \small Allauca Luiggy}
\bigskip

\vspace{.1 in}
\hrule
\makebox[3.5in][l]


{\leftskip=.6in  \parindent=-.3in  \parskip=.05in

\bigskip

\centerline {\namefont \LARGE \bf SCAPE PLAN}
\bigskip
\bigskip
\bigskip

{\bf DESCRIPCI\'ON DE LA APLICACI\'ON}
\bigskip
\\Esta aplicaci\'on de interfaz muy sencilla ha sido desarrollada en Python se trata de un juego para personas no videntes, este juego se trata de una historia donde el jugador va armando la historia a su conveniencia. El juego se trata de un profugo de la justicia que debe de tomar varias desiciones en el trayecto del juego; de las cuales podr\'ian ser buenas o malas desiciones, de estas desiciones que tome podr\'ia poner en riesgo su vida; solo se acciona las teclas para que se ejecute el juego porque es un juego para personas ciegas.
\bigskip
\bigskip

{\bf FUNCIONALIDADES DE LA APLICACI\'ON}
\bigskip
\\1. Esta aplicaci\'on se trata de un juego, que consiste en guiar a un reo por un camino que lo lleve a su libertad.\\

2.Apenas se ejecute la aplicaci\'on se empezar\'an a reproducir los sonidos.\\

3.Se mostrar\'a una imagen, que corresponde al logotipo del juego, el cual se llama SCAPE PLAN.\\

4. A medida que se van reproduciendo los sonidos el jugador tiene opción de escoger las alternativas para el reo, el cual tendr\'a que decidir la que crea m\'as conveniente para cumplir con el objetivo.\\

\bigskip
\bigskip

{\bf A QUIEN VA DIRIGIDA LA APLICACI\'ON}
\bigskip
\\La aplicaci\'on ha sido implementada para personas no videntes, las cuales tienen desarrollados su sentidos del tacto y del o\'ido, mediante los cu\'ales podr\'an interactuar con este juego y ayudar\'ian a ejercitar cada uno de los sentidos como son el tacto y el o\'ido. 
\bigskip
\bigskip
\newpage
{\bf OBSERVACIONES}
\bigskip
\\Desarrollar en Python por primera vez requiere de mucha paciencia, tener conocimiento de lenguaje C, tener mucho cuidado y poner atenci\'on el las identaciones de las l\'ineas de c\'odigo, ya que puede generar errores al momento de ejecutar la aplicaci\'on. Tambi\'en revisar la sintaxis correcta de las funciones y conservar la estructura esqueleto de la app y donde se deben definir las funcionalidades que se requiere implementar.   
\bigskip
\bigskip


{\bf CONCLUSIONES}
\bigskip
\\En este proyecto se aprendi\'o a programar bajo la librer\'ia pygame de python, esta librer\'ia nos ayud\'o en la implementaci\'on de los sonidos ya que el juego consistia para personas con discapacidades visuales de las cuales solo se les relato la historia para de esta manera la persona ciega arme poco a poco la historia del juego hasta que llegue a su fin.\\
\\Benefici\'o mucho la documentación en castellano, de esta manera se facilito el uso de las principales funcionalidades del lenguaje y permiti\'o un mejor entendimiento de la l\'ogica del juego.\\
\bigskip
\bigskip

{\bf EXPERIENCIAS}
\bigskip
\\El aprender un lenguaje nuevo que cubra las expectativas que uno tiene como programador junior fue todo un \'exito al poder estudiarla. En Python todo se comporta de una manera muy l\'ogica la cual proporciona una mejor implementaci\'on del programa, adem\'as al ser un lenguaje que el c\'odigo es interpretado con sangr\'ias, esto acostumbra a escribir de una manera m\'as elegante y clara.\\
\\La informaci\'on sobre Python en la web es mucha y hay tutoriales variados en varios idiomas, la comunidad Python es muy grande y
siempre dispuesta a dar una mano o a dar nuevas soluciones para distintas problem\'aticas.\\
\\La \'unica desventaja es que al ser de tipado din\'amico, puede traer malas costumbres al cambiar de tipo las variables dentro de un programa, que luego en lenguajes de tipado est\'atico traer\'an problemas.

\end{document}
