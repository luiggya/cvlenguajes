\documentstyle[11pt]{report}
%\setcounter{page}{6}

\oddsidemargin 0pt \evensidemargin 0pt
\topmargin=1.25in
\headheight 10pt \headsep 10pt \footheight 10pt \footskip 24pt
\textheight 10in \textwidth 6.5in \columnsep 10pt \columnseprule 0pt

\font\namefont=cmr10 scaled\magstep2
\voffset=-.75in
\parskip=.075in
\parindent=0in

\thispagestyle{empty}
\begin{document}
\bigskip



\bigskip
\large \centerline {\namefont \bf BUILDING ID USER MANUAL} 
\bigskip

\centerline{\namefont  \small Luiggy Allauca}
\bigskip

\vspace{.1 in}
\hrule
\makebox[3.5in][l]


{\leftskip=.6in  \parindent=-.3in  \parskip=.05in

\bigskip

\centerline {\namefont \LARGE \bf Android}
\bigskip
\bigskip
\bigskip

{\bf 1 Introduction}
\bigskip
\\At the moment, there is much talk of QR codes, these codes are a open source of information storage in dimensional format. The three squares were in your corners were used to adjust its position, thus the software can interpret and decode the information. The information is converted into points called pixels, the pixels can see by camera that we use for check them.\\
\\This manual has been created to facilitate the use of the building Id application, it was elaborated with Android OS, the same that was proposed as the first project of Programming Languages asignature in Term II 2012
\bigskip
\bigskip

{\bf 2. Requeriments}
\bigskip
\\1. 	An Android device, superior o equal version to 2.3 that can executed a mobile application.\\
\\2.	An mobile phone has Android SO and camera.

\bigskip
\bigskip

{\bf 3. The Application}
\bigskip
\\a. Name and description of Project
\\Building Id is an application for Android devices. This application can allowed to know the information about a place specific. The places can be Faculties, canteens, etc. This is possible by reading to QR codes that they find visible and access easy in the buildings.\\
\\Once the reading of QR codes, you can see the information about a building, automatically the image saved on the mobile phone for you can search later without you need to scan the code again. Thus, the application does a search to Database and show the information all building that were previously searched.\\
\\b. Features
\\1. Using the device camera.
\\2. Database storage.
\\3. Classification by Category (all, building and canteens)\\
\\c. Functionalities
\\Building Id has a main menu with the following options:
\\1. Read QR codes
\\2. Show buildings
\\3. Search and
\\4. Exit\\
\bigskip
\bigskip
{\bf  Read QR codes}
\bigskip
\\The option “Read QR codes” will allow to read codes by camera.\\
\bigskip
\bigskip
{\bf  Show Buildings}
\bigskip
\\The option Show buildings will see to information about places that they are stored in the database. This information is categorized by buildings or canteens using tags for a better appearance.\\
\bigskip
\bigskip
{\bf Search}
\bigskip
\\The option Search will allow to doing a searh of places that they stored in the database. When you write the first letter begins to display a list of faculties or canteens found in the phone database.\\
\bigskip
\bigskip
{\bf EXIT}
\bigskip
\\The option Exit abort the application in the case that you had log in Building ID by error or you want to exit the application



\end{document}
