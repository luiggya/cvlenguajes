\documentclass[a4paper,11pt]{article}

\usepackage[latin1]{inputenc}
\usepackage{color}
\usepackage{array}
\usepackage[Spanish]{babel}
\usepackage{amsmath,amssymb}
\usepackage{graphicx}
\usepackage{ragged2e}

\addtolength{\textwidth}{2cm}
\addtolength{\hoffset}{-1cm}




\title{``Building Id''}
\author{Tania S\'{a}nchez\\  Isaac Torres  \\Luiggy Allauca}

\begin{document}

\setlength{\topmargin}{0.5in}
			\pagestyle{empty}
			\begin{center}
				\textbf{
					\vspace{-0.7em}
					ESCUELA SUPERIOR POLIT\'{E}CNICA DEL LITORAL
				}
				\line(1,0){380}\\		
				\scriptsize{FACULTAD DE INGENIER\'{I}A EN ELECTRICIDAD Y COMPUTACI\'{O}N}
				
				\vspace{2.5em}
				\includegraphics[scale=0.3]{image/espol.jpg} 
			\end{center}
			
			\begin{center}
				\vspace{2.5em}
				Lenguajes de Programaci\'{o}n
				\\II T\'{e}rmino - 2012
				\vspace{1.5em}  %espacio entre linea anterior y la siguiente
				\\Manual de Usuario \\
				\vspace{5em}
				\Huge{\textbf{``Building Id''	\vspace{3em}}}
			\end{center}	
			
			


				\hspace*{5cm}Desarrolladores:
				\vspace{1.5em}
				\\\hspace*{8cm}Tania S\'{a}nchez
				\\\hspace*{8cm}Isaac Torres
				\\\hspace*{8cm}Luiggy Allauca


\newpage 

\setlength{\topmargin}{-0,2in}
\tableofcontents

\newpage 


\section{Introducci\'{o}n}
En la actualidad se habla mucho de los c\'{o}digos QR, estos c\'{o}digos son un sistema abierto de almacenamiento de informaci\'{o}n en formato bidimensional. Los tres cuadrados que est\'{a}n en sus esquinas permiten regular su posici\'{o}n, de esta manera el software que los interpreta puede encontrar la informaci\'{o}n a decodificar. La informaci\'{o}n est\'{a} transformada en puntos (pixeles) reconocibles por la c\'{a}mara que se usa para chequearlos.
\newline
\newline
El presente manual ha sido creado para facilitar el uso de la aplicaci\'{o}n Building Id, elaborado en el sistema operativo Android, la misma que fue propuesta como primer proyecto de la materia de Lenguajes de Programaci\'{o}n en el a\H{n}o 2012 t\'{e}rmino II.
\newline
\newline
\section{Requisitos}
Un dispositivo Android con versi\'{o}n mayor o igual a la 2.3 que vaya a ejecutar una aplicaci\'{o}n m\'{o}vil. 
\newline
Un tel\'{e}fono m\'{o}vil con c\'{a}mara que albergue el sistema operativo Android.
\newline
\newline
\section{La aplicaci\'{o}n}

\subsection{Nombre y Descripci\'{o}n del Proyecto}
\textbf{\textit{ ``Building Id''}}
Es una aplicaci\'{o}n para dispositivos Android. Esta aplicaci\'{o}n nos permite saber la infomaci\'{o}n de un lugar determinado. Lugares tales como Facultades, Comedores, etc, esto es posible mediante la lectura de c\'{o}digos QR que estar\'{a}n de manera visible y de f\'{a}cil acceso en estos edificios. 
\newline
\newline
Una vez realizada la lectura del c\'{o}digo QR se muestra la informaci\'{o}n del edificio correspondiente, automaticamente se guarda en el tel\'{e}fono para posteriormente ser consultada sin la necesidad del c\'{o}digo nuevamente, permitiendo as\'{i} realizar una \newline busqueda en la base y ver esta informaci\'{o}n de los diferentes lugares.
\newline
\newline
\subsection{Caracter\'{i}sticas}
\begin{itemize}
\item 
Uso de la c\'{a}mara del dispositivo
\item
Almacenamiento en base de datos
\item
Clasificaci\'{o}n por categor\'{i}as. (Todos, Edificio y Comedor)
\end{itemize}


\newpage

\subsection{Funcionalidades}

\textbf{\textit{``Building Id''}} cuenta con un men\'{u} inicio donde tenemos las opciones de: 
\newline\newline
\hspace*{3cm}Leer c\'{o}digo QR,
\newline 
\hspace*{3cm}Ver edificios,
\newline 
\hspace*{3cm}Buscar y 
\newline
\hspace*{3cm}Salir.
\begin{center}
	\includegraphics[scale=0.37]{image/menu.png}
\end{center}
	
\subsubsection{Leer c\'{o}digo QR}

La opci\'{o}n \textbf{``Leer c\'{o}digo QR''} permitir\'{a} la lectura del c\'{o}digo QR por medio de la c\'{a}mara.
\begin{center}
		\includegraphics[scale=0.2]{image/codigo_qr.jpg}
	\end{center}

Una vez leido el c\'{o}digo nos muestra una galer\'{i}a de imagenes y su respectiva informaci\'{o}n.
		\\
		\begin{center}
				\includegraphics[scale=0.43]{image/galeriafict.png}
		\end{center}

\newpage
\hspace*{2cm}\textbf{Informaci\'{o}n General y Estructura Interna} 
		\newline		
		\hspace*{4cm}\textit{``en este caso de FICT''}
		\newline
		
		
		\begin{figure}[h] % indico que voy a poner una figura y [h] indica que la posición relativa, tambien puedo usar t = top entre otros.
		\hfill
			\begin{minipage}[t]{.45\textwidth}
				\begin{center}
					\includegraphics[scale=0.41]{image/informacionfict.PNG} 
				\end{center}
			\end{minipage}
			\hfill
			\begin{minipage}[t]{.45\textwidth}
				\begin{center}
					\includegraphics[scale=0.3]{image/estructuraFict.png} 
				\end{center}
			\end{minipage}
		\hfill
		\end{figure}
	
	
	
	\subsubsection{Ver Edificios}
	
		La opci\'{o}n de \textbf{``Ver Edificios''} nos muestra la infomaci\'{o}n de los lugares que est\'{a}n almacenados en la base.
		\newline
		\newline
		Categorizada en Edificio o Facultades y Comedores con la utilizaci\'{o}n de tags para una mejor apariencia.
		
		\begin{center}
			\includegraphics[scale=0.5]{image/veredificios.png}
		\end{center}
	
	
		
\newpage

\subsubsection{Buscar}

			 La opci\'{o}n \textbf{``Buscar''} permitir\'{a} una busqueda de los lugares que est\'{a}n almacenados en la base.
			
			Al momento de escribir la primera letra empieza a mostrar una lista de las facultades o comedores que se encuentran en la base de datos del tel\'{e}fono.
			\newline
		
		
		
		\begin{figure}[h] % indico que voy a poner una figura y [h] indica que la posición relativa, tambien puedo usar t = top entre otros.
		\hfill
			\begin{minipage}[t]{.45\textwidth}
				\begin{center}
					\includegraphics[scale=0.33]{image/busqueda.PNG} 
				\end{center}
			\end{minipage}
			\hfill
			\begin{minipage}[t]{.45\textwidth}
				\begin{center}
					\includegraphics[scale=0.3]{image/autocompletarfict.png} 
				\end{center}
			\end{minipage}
		\hfill
		\end{figure}



\subsubsection{Salir}

		La opci\'{o}n \textbf{``Salir''} est\'{a} en caso de que se haya entrado por error o simplemente ya no se quiera usar la aplicaci\'{o}n. base.
		\newline



\newpage

\addcontentsline{toc}{section}{Referencias}
\begin{thebibliography}{99}
\bibitem{unistan}http://androideity.com.
\bibitem{acp} http://www.javaya.com.ar/androidya.
\end{thebibliography}


\end{document}