\documentclass[a4paper,11pt]{article}

\usepackage[latin1]{inputenc}
\usepackage{color}
\usepackage{array}
\usepackage[Spanish]{babel}
\usepackage{amsmath,amssymb}
\usepackage{graphicx}
\usepackage{ragged2e}

\addtolength{\textwidth}{2cm}
\addtolength{\hoffset}{-1cm}




\title{``Building Id''}
\author{Tania S\'{a}nchez\\  Isaac Torres  \\Luiggy Allauca}

\begin{document}


\setlength{\topmargin}{0.5in}
			\pagestyle{empty}
			\begin{center}
				\textbf{
					\vspace{-0.7em}
					ESCUELA SUPERIOR POLIT\'{E}CNICA DEL LITORAL
				}
				\line(1,0){380}\\		
				\scriptsize{FACULTAD DE INGENIER\'{I}A EN ELECTRICIDAD Y COMPUTACI\'{O}N}
				
				\vspace{2.5em}
				\includegraphics[scale=0.3]{image/espol.jpg} 
			\end{center}
			
			\begin{center}
				\vspace{2.5em}
				Lenguajes de Programaci\'{o}n
				\\II T\'{e}rmino - 2012
				\vspace{1.5em}  %espacio entre linea anterior y la siguiente
				\\Observaciones, conclusiones y experiencias \\
				\vspace{5em}
				\Huge{\textbf{``Building Id''	\vspace{3em}}}
			\end{center}	
			
			


				\hspace*{5cm}Desarrolladores:
				\vspace{1.5em}
				\\\hspace*{8cm}Tania S\'{a}nchez
				\\\hspace*{8cm}Isaac Torres
				\\\hspace*{8cm}Luiggy Allauca


\newpage 

\setlength{\topmargin}{-0,2in}
\tableofcontents

\newpage 


\section{Observaciones}
En esta aplicaci\'{o}n se produjo algunas variantes para tener su versi\'{o}n final, las pantallas fueron un problema al saber elegir bien un estilo de dise\H{n}o que se adapte a la aplicaci\'{o}n que se estaba implementando, en primera instancia se escogi\'{o} un patr\'{o}n de dise\H{n}o que se lo hab\'{i}a hecho en photoshop pero no brindaba las garant\'{i}as necesarias para poder dise\H{n}ar algunas funcionalidades de la aplicaci\'{o}n como por ejemplo no se lograba tener un bot\'{o}n que se adapte al dise\H{n}o que se lo hab\'{i}a hecho en photoshop. Lo que se hizo fue implementar los estilos de dise\H{n}o para la aplicaci\'{o}n, donde se graduaba los colores y efectos que los botones brindaban al hacer un clic sobre el mismo.

Los c\'{o}digos QR fueron los que m\'{a}s tiempo llevo implementarlos porque hab\'{i}a que adaptarlo a nuestro proyecto, las librer\'{i}as nos facilitaron el trabajo para que ejecutar\'{a} la captaci\'{o}n del patr\'{o}n de dise\H{n}o por medio de la c\'{a}mara del dispositivo m\'{o}vil. Como nuestro proyecto se trataba de identificar algunos edificios de la ESPOL, se tuvo que dise\H{n}ar algunos patrones para que se identificara a cada edificio por su patr\'{o}n de dise\H{n}o asignado.
\newline
\newline
\section{Experiencia del desarrollo}
Desarrollar en android represent\'{o} un reto al principio, ya que el estilo de programaci\'{o}n es diferente a lo normalmente acostumbrado, sin embargo fue solo cuesti\'{o}n de tiempo adaptarse. Otro punto llamativo fue la interacci\'{o}n con el usuario, la cual es diferente a la cl\'{a}sica. Considero que a pesar de ciertas dificultades, la experiencia fue productiva, ya que el desarrollo para dispositivos m\'{o}viles est\'{a} en pleno auge. 
\newline
\newline
\section{Conclusiones}
En este proyecto se implementaron los c\'{o}digos QR los cuales nos produjeron una respuesta r\'{a}pida y concisa del lugar que se deseaba la informaci\'{o}n, en nuestro caso la informaci\'{o}n de los edificios y comedores de la ESPOL.

Se elabor\'{o} una galer\'{i}a de los diferentes edificios y comedores que se tuvieron en el proyecto, aqu\'{i} se presentaron fotos referente al edificio o comedor adem\'{a}s se mostr\'{o} informaci\'{o}n acerca de la estructura interna de cada uno de ellos; por ejemplo las aulas, los laboratorios, los auditorios y las oficinas de los profesores.

}
\end{document}